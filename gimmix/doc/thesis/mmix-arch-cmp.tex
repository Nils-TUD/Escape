\section{Comparisons}

MMIX has four instructions to compare numbers, which for example can be used for branching. Additionally, it has instructions to conditionally set and zero or set a register.

\instrtbl
	{\mi{CMP \$X,\$Y,\$Z|Z}}
	{$\dr{X} \leftarrow (s(\dr{Y}) > \sdrimm{Z}) - (s(\dr{Y}) < \sdrimm{Z})$}
\noindent The instruction \mi{CMP} compares \dr{Y} with \udrim{Z} using signed arithmetic and puts the result into \dr{X}. If \dr{Y} is less than \udrim{Z}, \dr{X} will be set to $-1$, if they are equal, \dr{X} will be set to 0 and if \dr{Y} is greater than \udrim{Z}, \dr{X} will be set to 1. \citep[pg. 11]{mmix-doc}

\instrtbl
	{\mi{CMPU \$X,\$Y,\$Z|Z}}
	{$\dr{X} \leftarrow (\dr{Y} > \udrim{Z}) - (\dr{Y} < \udrim{Z})$}
\noindent This instruction behaves like \mi{CMPU}, but uses unsigned arithmetic. \citep[pg. 11]{mmix-doc}

\instrtbl
	{\mi{CSN|CSZ|CSP|CSOD|CSNN|CSNZ|CSNP|CSEV \$X,\$Y,\$Z|Z}}
	{if $s(\dr{Y}) <0|=0|>0|odd|\ge0|\ne0|\le0|even$: $\dr{X} \leftarrow \udrim{Z}$}
\noindent The family of conditional set instructions will set \dr{X} to \udrim{Z}, if \dr{Y} is negative, zero, positive, odd, nonnegative, nonzero, nonpositive or even. Otherwise nothing happens. \citep[pg. 11]{mmix-doc}

\instrtbl
	{\mi{ZSN|ZSZ|ZSP|ZSOD|ZSNN|ZSNZ|ZSNP|ZSEV \$X,\$Y,\$Z|Z}}
	{$\dr{X} \leftarrow (s(\dr{Y}) <0|=0|>0|odd|\ge0|\ne0|\le0|even)~?~\udrim{Z}~:~0$}
\noindent Very similar to the conditional set instructions, the zero or set instructions set \dr{X} either to \udrim{Z} or zero, depending on whether \dr{Y} is negative, zero, positive, odd, nonnegative, nonzero, nonpositive or even. \citep[pg. 11]{mmix-doc}

\medskip

MMIX does also provide an atomic \i{compare and swap} instruction. It can be used for interprocess communication with shared memory or to synchronize threads in the same process. Since MMIX is not only designed to work on a single processor, this instruction might also be helpful when independent computers are sharing the same memory. \citep[pg. 25]{mmix-doc}

\instrtblfour
	{\mi{CSWAP \$X,\$Y,\$Z|Z}}
	{if $\vmem{8}{\dr{Y} + \udrim{Z}}~= \sr{P}$:}
	{$\quad \vmem{8}{\dr{Y} + \udrim{Z}}~\leftarrow \dr{X},\quad \dr{X} \leftarrow 1$}
	{else:}
	{$\quad \sr{P} \leftarrow \vmem{8}{\dr{Y} + \udrim{Z}},\quad \dr{X} \leftarrow 0$}
\noindent The \i{compare and swap octabytes} instruction compares \vmem{8}{\dr{Y} + \udrim{Z}} with the special \i{prediction register} \sr{P} and either replaces the octa in memory with \dr{X} or \sr{P} with the octa in memory, depending on whether \sr{P} is equal to the octa. \dr{X} indicates whether the octa in memory has been replaced. \citep[pg. 25]{mmix-doc} For example, one could set \dr{0} to 1, \sr{P} to 0 and do a \mi{CSWAP \$0,\$Y,\$Z|Z}, assuming that $\dr{Y} + \udrim{Z}$ denotes the memory location that is desired for synchronization. If \dr{0} has been set to 1, the lock has been aquired successfully. If not, the whole procedure will be repeated. That means, \vmem{8}{\dr{Y} + \udrim{Z}} being 1 or 0 would indicate that someone currently has the lock or not, respectively.

